%%%%%%%%%%%%%%%%%%%%%%%%%%%%%%%%%%%%%%%%%
% Medium Length Professional CV
% LaTeX Template
% Version 2.0 (8/5/13)
%
% This template has been downloaded from:
% http://www.LaTeXTemplates.com
%
% Original author:
% Trey Hunner (http://www.treyhunner.com/)
%
% Important note:
% This template requires the resume.cls file to be in the same directory as the
% .tex file. The resume.cls file provides the resume style used for structuring the
% document.
%
%%%%%%%%%%%%%%%%%%%%%%%%%%%%%%%%%%%%%%%%%

%----------------------------------------------------------------------------------------
%	PACKAGES AND OTHER DOCUMENT CONFIGURATIONS
%----------------------------------------------------------------------------------------

\documentclass{resume} % Use the custom resume.cls style

\usepackage[left=0.7in,top=0.7in,right=0.7in,bottom=0.7in]{geometry} % Document margins
\usepackage{soul} % \ul function for underlining with line breaks

\name{Kathryn B. Newhart} % Your name
\address{646 Swift Road, West Point, New York 10996} % Your address
\address{kathryn.newhart@westpoint.edu} % Your phone number and email

\begin{document}

%----------------------------------------------------------------------------------------
%	WORK EXPERIENCE SECTION
%----------------------------------------------------------------------------------------

\begin{rSection}{Professional Experience}

\begin{rSubsection}{United States Military Academy at West Point}{June 2021 - Present}{Assistant Professor of Environmental Engineering}{West Point, NY}
\end{rSubsection}

%------------------------------------------------

\begin{rSubsection}{Metro Wastewater Reclamation District}{March 2020 - May 2021}{Technology \& Innovation Engineer Associate}{Denver, CO}
\end{rSubsection}

%------------------------------------------------

\begin{rSubsection}{Red Rocks Community College}{August 2020 - December 2020}{Water Quality Management Instructor}{Lakewood, CO}
\end{rSubsection}

%------------------------------------------------

\begin{rSubsection}{Colorado School of Mines}{May 2016 - May 2019}{Teaching Assistant}{Golden, CO}
\end{rSubsection}

%----------------------------------------------------------------------------------------
%	TEACHING SECTION
%----------------------------------------------------------------------------------------

%\begin{rSection}{Teaching}
\\

\begin{table}[h]
\centering
\textbf{Teaching experience}
\begin{tabular}{lllc}
\toprule
  \emph{Institution} & \emph{Course} & \emph{Title (Credit Hours)} & \emph{Semesters}\\
\midrule			
  USMA & EV450 & Environmental Engineering for Sustainable Development (3) & 1 \\
  USMA & EV490/491 & Environmental Engineering Design (3) & 2  \\
  USMA & EV201\textsuperscript{1}& Introduction to Environmental Engineering (3) & -  \\
  USMA & EV401\textsuperscript{1} & Physical and Chemical Treatment (3.5) & 1  \\
  RRCC & WQM42\textsuperscript{1} & Water Data Management \& Analysis (3) & 1 \\
  CSM & CEE 470/570\textsuperscript{2} & Unit Processes for Water and Wastewater Treatment (3) & 3 \\
  CSM & CEE 471/571\textsuperscript{2} & Advanced Water Treatment and Reclamation (3) & 1 \\
  CSM & CEE 330\textsuperscript{2} & Field Session for Environmental Engineering (3) & 3  \\
\bottomrule
\end{tabular}

\smallskip\footnotesize{\textsuperscript{1} Course director, \textsuperscript{2} TA / Instructor role}

\end{table}
%{\footnotesize \textsuperscript{1} Course director, \textsuperscript{2} TA / Instructor role}

%\end{rSection}

\end{rSection}

%----------------------------------------------------------------------------------------
%	EDUCATION SECTION
%----------------------------------------------------------------------------------------

\begin{rSection}{Education}

\begin{rSubsection}{Doctor of Philosophy}{2018 - 2020}{Civil and Environmental Engineering}{Colorado School of Mines, Golden, CO}
\item Dissertation: “Data-driven process control of municipal wastewater treatment”
\item Advisors: Prof. Tzahi Cath and Prof. Amanda Hering (Baylor University)
\end{rSubsection}

\begin{rSubsection}{Master of Science}{2016 - 2018}{Civil and Environmental Engineering}{Colorado School of Mines, Golden, CO}
\end{rSubsection}

\begin{rSubsection}{Bachelor of Science}{2013 - 2016}{Environmental Engineering}{Colorado School of Mines, Golden, CO}
\end{rSubsection}

\end{rSection}


%----------------------------------------------------------------------------------------
%	PUBLICATION SECTION
%----------------------------------------------------------------------------------------

\begin{rSection}{Publications}
\begin{etaremune}
	\item \textbf{Newhart, K.B.}, Hering, A.S., Cath, T.Y., ``Data science tools to enable decarbonized water and wastewater treatment systems.'' \emph{Pathways to Water Sector Decarbonization, Carbon Capture and Utilization}, edited by Z. Jason Ren and Krishna Pagilla, IWA Publishing, 2022.

	\item \textbf{Newhart, K.B.}, Goldman-Torres, J., Wisdom, B. Freedman, D., Hering, A.S., Cath, T.Y., ``Real-time dose control of peracetic acid disinfection in municipal wastewater treatment,'' \emph{ACS EST Water}, 2021, 1, 2, 328–338

	\item \textbf{Newhart, K.B.}, Marks, C.A., Rauch-Williams, T., Cath, T.Y., Hering, A.S. ``Hybrid statistical-machine learning ammonia forecasting in continuous activated sludge treatment for improved process control,'' \emph{Journal of Water Process Engineering}, 2020, 37, 101389

	\item Klanderman, M., \textbf{Newhart, K.B.}, Cath. T.Y., Hering, A.S., ``Fault isolation for a complex decentralized wastewater treatment facility,'' \emph{Journal of the Royal Statistical Society}, Series C., 2020, 69, 931-951.

	\item \textbf{Newhart, K.B.}., Holloway, R.W., Hering, A.S., Cath, T.Y., ``Data-driven performance analyses of wastewater treatment plants: A review,'' \emph{Water Research}, 2019, 157, 498-513

	\item Odom, G.J., \textbf{Newhart, K.B.}, Cath, T.Y., Hering, A.S., ``Multi-state multivariate statistical process control,'' \emph{Applied Stochastic Models in Business and Industry}, 2018, 34(6), 880-892

	\item Bell, E.A., Poynor, T.E., \textbf{Newhart, K.B.}, Regnery, J., Coday, B.D., Cath, T.Y., ``Produced water treatment using forward osmosis membranes: evaluation of extended-time performance and fouling,'' \emph{Journal of Membrane Science}, 2017, 525, 77-88.
\end{etaremune}

\end{rSection}


%----------------------------------------------------------------------------------------
%	RESEARCH SECTION
%----------------------------------------------------------------------------------------

\begin{rSection}{Research}
\emph{Note: Federal law heavily restricts research funding from outside agencies. Select federal funding agencies do not allow federal employees to be listed as PI's or co-PI's (e.g., US EPA). Therefore, PI is noted when scope is performed and participation is noted when only support is provided.}

``Crossing the Finish Line: Integration of Data-Driven Process Control for Maximization of Energy and Resource Efficiency in Advanced Water Resource Recovery Facilities,'' U.S. Department of Energy, Research and Development for Advanced Water Resource Recovery Systems. DE-FOA-0002336. Awarded 2021. Total award \$2,400,000. \ul{Principal Investigator at West Point}.

``Data-driven Fault Detection and Process Control for Potable Reuse with Reverse Osmosis,'' National Alliance for Water Innovation, Autonomous Water and Precision Separations. NAWI-2-2021. Awarded 2021. \ul{Principal Investigator at West Point}.
\end{rSection}


%----------------------------------------------------------------------------------------
%	CONFERENCE SECTION
%----------------------------------------------------------------------------------------

\begin{rSection}{Select Conference Presentations}

``Predictive Control in Wastewater Treatment Facilities Using Simple Statistical Models,'' South Platte Coalition for Urban River Evaluation: Confluence at the Confluence, Oct 15, 2019, Englewood, CO

``Energy Reduction in Municipal Wastewater Treatment,'' Colorado Industrial Pretreatment Coordinators Association Fall Conference, Oct 18, 2019, Black Hawk, CO

%``Ammonia Forecasting Using Statistical Models of Activated Sludge Treatment,'' Oct 3, 2019, Re-inventing the Nation’s Urban Water Infrastructure (ReNUWIt) NSF ERC Industrial Advisory Board Meeting, Golden, CO
%
``Predictive Modelling and Performance Assessment of Ammonia-Based Aeration Control,'' Water Environment Federation Technical Exhibition and Conference (WEFTEC), Sept 23, 2019, Chicago, IL

``A Utility Perspective: Practical Considerations of Operating and Advancing Ammonia-Based Aeration Control,'' July 10, 2019, RMWEA Innovation Seminar, Denver, CO

``Fault Detection Using PCA at a Municipal Wastewater Treatment Facility,'' July 30, 2019, Joint Statistical Meeting, Denver, CO

%``Use of Statistical Process Control at a Decentralized Water Reclamation Facility in Colorado,'' Colorado School of Mines Graduate Research and Discovery Symposium, April 18, 2019, Golden, CO
%
%``Statistical Process Control in Municipal Wastewater Treatment,'' RMSAWWA/RMWEA Joint Annual Conference, Sept 13, 2018, Denver, CO
%
%“Advanced Statistical Modeling of a Pilot-Scale Biological Wastewater Treatment System for Fault Detection,” RMSAWWA/RMWEA Student Conference, May 14, 2018, Golden, CO
%
%“Principal Component Analysis for Monitoring of Biological and Membrane Wastewater Treatment Systems,” Colorado School of Mines Graduate Research and Discovery Symposium, April 5, 2018, Golden, CO
%
“Performance Evaluation of a Sequencing Batch Membrane Bioreactor Using Principal Component Analysis,” Annual WateReuse Symposium, Sept 11, 2017, Phoenix, AZ

“Use of Principal Component Analysis for Early-Fault Detection in a Pilot-Scale Biological Wastewater Treatment System,” Quality and Productivity Research Conference, June 14, 2017, Storrs, CT

%“Statistical Process Control for Monitoring Biological Wastewater Treatment Systems,” RMSAWWA/RMWEA Student Conference, May 22, 2017, Albuquerque, NM
%
%Newhart, K.B. & Avila, I., NDMA: relevance and regulatory status for drinking water facilities, Rocky Mountain Water, November 2017
%
\end{rSection}

%----------------------------------------------------------------------------------------
%	WORKSHOP SECTION
%----------------------------------------------------------------------------------------

\begin{rSection}{Workshop Organization and Partiticpation}

``Visualization, Analysis, and Modeling in R for the Water Professional'' \empth{MoWaTER PRO: Data Science Workshop}, December 2021, Develop, organize, and present.

``Machine Learning in the Water Industry'' \emph{WEF Innovations in Process Engineering}, June 8, 2021, Organize and present

``A Hypothetical – Potable Reuse Moves Towards Artificial Intelligence,'' \emph{36th Annual WateReuse Symposium}, March 1, 2021, Panelist

``Understanding and Embracing Machine Learning, Artificial Intelligence and Predictive Analytics,'' \emph{AWWA/SWAN International Smart Water Symposium}, November 10, 2020, Facilitator and presenter

``Data Research Advances Water Industry,'' \emph{NSF Mid-scale Research Infrastructure Workshop for Intelligent Water Systems}, August 25, 2020, Virtual, Facilitator and presenter

\end{rSection}

\newpage

%----------------------------------------------------------------------------------------
%	LEADERSHIP SECTION
%----------------------------------------------------------------------------------------

\begin{rSection}{Leadership and Service Roles}

Technology Reviewer, Water Research Foundation TechLink, January 2022 – present

Referee: ACS Environmental Science \& Technology Engineering; Environmental Science: Water Research \& Technology; Resources, Conservation \& Recycling, Water Environment Research

Department representative, Superitendents Civilian Faculty Advisory Council, USMA, January 2022 – present

Member, AWWA Water Science \& Research Division, Information Management \& Technology, 2021 – present

President, NSF ReNUWIt Engineering Research Center Student Leadership Committee, 2018 – 2019

President, CSM Campus Chapter of the Rocky Mountain Section of the American Water Works Association (RMSAWWA)/Rocky Mountain Water Environment Association (RMWEA), 2018 – 2019

Co-Chair, 15th Annual RMSAWWA/RMWEA Joint Student Conference, 2018

\end{rSection}

%----------------------------------------------------------------------------------------
%	PROFESSIONAL MEMBERSHIPS
%----------------------------------------------------------------------------------------
\begin{rSection}{Professional Memberships}

American Chemical Society

Water Environment Federation

\end{rSection}

%----------------------------------------------------------------------------------------
%	IN THE NEWS SECTION
%----------------------------------------------------------------------------------------
\begin{rSection}{In The News}

Newhart, K. B., Marks, C., Rauch-Williams, T., Cath, T. Y., Hering, A. S. (2020) “Boulder tests its waters with predictive aeration control,” Advances in Water Research, 30: 25–28. \href{https://www.advancesinwaterresearch.org/awr/20200709/MobilePagedArticle.action?articleId=1621836&pm=1#articleId1621836}{URL}.
\end{rSection}

%----------------------------------------------------------------------------------------
%	AWARDS SECTION
%----------------------------------------------------------------------------------------

\begin{rSection}{Awards}

WEF/WRF LIFT Intelligent Water System Challenge, 1st place, 2019

AWRA-Colorado Rich Herbert Memorial Scholarship, 2019

\end{rSection}

%----------------------------------------------------------------------------------------
%	CERTIFICATIONS SECTION
%----------------------------------------------------------------------------------------

\begin{rSection}{Certifications}

Wastewater Operator, Class D, Colorado, 2016-2024

Fundamentals of Engineering (FE), Environmental, Colorado, NCEES ID 16-475-7
\end{rSection}



\end{document}
